\chapter{Conclusion}
\label{ch:sum}
\section{Summary of Findings}
This thesis presented a real-time, map-based localization system that combines LiDAR-Inertial Odometry (FAST-LIO2), multithreaded Normal Distributions Transform (NDT) scan matching, and a fixed-lag factor graph optimization framework. The system was designed to provide accurate and drift-resilient 6-DoF pose estimation in large-scale, GNSS-denied, and partially dynamic environments.

Evaluations conducted on benchmark and custom datasets demonstrated that the system consistently achieved centimeter- to decimeter-level translational RMSE, outperforming standalone odometry methods such as FAST-LIO2, which exhibited meter-level drift in long trajectories.The proposed method achieved similar or superior localization
accuracy compared to recent map-based approaches.Furthermore,compared to SLAM-based techniques that rely on loop closures, the proposed method maintained better global consistency by continuously aligning LiDAR scans with a pre-built 3D map.

Real-time performance was sustained with an average processing latency of under 23 ms per frame, enabled by a multithreaded NDT implementation and dynamic submap loading. This ensured scalability across large environments while maintaining memory and computational efficiency. Additionally, dynamic object removal using a deep learning-based detection model improved registration stability in semi-dynamic scenes, and NDT demonstrated resilience under moderate visibility degradation such as fog.
The system also maintained robust localization in feature-sparse map boundaries and unmapped transition zones, where scan-to-map registration typically fails. By relying on high-frequency odometry within the factor graph, it preserved pose continuity and minimized drift.

These findings confirm that the proposed localization pipeline is well-suited for high-accuracy, real-time navigation in complex, unstructured, and dynamic environments.

%6.2 Research Questions Revisited
%
%This section revisits the key research questions outlined in the introduction and summarizes how they were addressed throughout the thesis:
%
%RQ1: How can LiDAR-Inertial Odometry (FAST-LIO2) and NDT-based map matching be effectively fused to enhance real-time localization accuracy and robustness?
%→ The fusion was achieved through a sliding-window factor graph optimization framework, which combined the high-frequency motion estimates from FAST-LIO2 with global pose corrections from NDT scan matching. This integration resulted in both high accuracy and drift resilience.
%
%RQ2: How does integrating a prior map improve the accuracy and robustness of LiDAR-based localization in GNSS-challenged environments?
%→ By registering live LiDAR scans to a pre-built 3D map, the system maintained global alignment and reduced drift significantly, even in the absence of loop closures or GNSS signals.
%
%RQ3: How does the proposed system maintain real-time performance and scalability in large-scale environments?
%→ The use of multithreaded NDT-OMP and a grid-based submap loading strategy enabled efficient scan-to-map matching, keeping per-frame processing time under 23 ms and limiting memory usage.
%
%RQ4: Does the proposed localization method maintain accuracy in feature-sparse and noisy environmental conditions?
%→ Yes. The system preserved localization accuracy in challenging scenarios, including feature-deprived zones and foggy conditions. Dynamic object removal further improved the robustness of scan registration.
%
%Through these contributions, the thesis provides a comprehensive and extensible solution to the problem of robust, real-time localization in complex environments, and establishes a strong foundation for future integration with additional sensing modalities such as camera or radar.
%
%
%\section{Conclusion}
%
%\subsection{6.1 Summary of Findings}
%
%This thesis presented a real-time, map-based localization system that combines LiDAR-Inertial Odometry (FAST-LIO2), multithreaded Normal Distributions Transform (NDT) scan matching, and a fixed-lag factor graph optimization framework. The system was designed to provide accurate and drift-resilient 6-DoF pose estimation in large-scale, GNSS-denied, and partially dynamic environments.
%
%Evaluations conducted on benchmark and custom datasets demonstrated that the system consistently achieved centimeter- to decimeter-level translational RMSE, outperforming standalone odometry methods such as FAST-LIO2, which exhibited meter-level drift in long trajectories. Compared to SLAM-based techniques that rely on loop closures, the proposed method maintained better global consistency by continuously aligning LiDAR scans with a pre-built 3D map.
%
%Furthermore, when tested on benchmark datasets such as KITTI Sequence 05 and MulRan KiAST sequences, the proposed method achieved similar or superior localization accuracy compared to recent map-based approaches including KISS-ICP and MOLA SLAM. While these systems depend on loop closures to correct drift, the proposed pipeline maintains global consistency in an open-loop manner, leveraging prior map alignment and tightly fused odometry without requiring revisits.
%
%Real-time performance was sustained with an average processing latency of under 23 ms per frame, enabled by a multithreaded NDT implementation and dynamic submap loading. This ensured scalability across large environments while maintaining memory and computational efficiency. Additionally, dynamic object removal using a deep learning-based detection model improved registration stability in semi-dynamic scenes, and NDT demonstrated resilience under moderate visibility degradation such as fog.
%
%Importantly, the system also maintained robust localization performance in challenging scenarios involving feature-sparse map boundaries and unmapped transition zones. In such cases—where scan-to-map registration often fails due to limited geometric information—the system successfully leveraged high-frequency odometry propagation within the factor graph to preserve continuity and minimize drift.
%
%These findings confirm that the proposed localization pipeline is a practical, extensible, and accurate solution for autonomous systems operating in complex, unstructured, and dynamic environments.
%
%\subsection{6.2 Research Questions Revisited}
%
%This section revisits the key research questions outlined in the introduction and summarizes how they were addressed throughout the thesis:
%
%\begin{itemize}
%	\item \textbf{RQ1:} How can LiDAR-Inertial Odometry (FAST-LIO2) and NDT-based map matching be effectively fused to enhance real-time localization accuracy and robustness?  
%	→ The fusion was achieved through a sliding-window factor graph optimization framework, which combined the high-frequency motion estimates from FAST-LIO2 with global pose corrections from NDT scan matching. This integration resulted in both high accuracy and drift resilience.
%	
%	\item \textbf{RQ2:} How does integrating a prior map improve the accuracy and robustness of LiDAR-based localization in GNSS-challenged environments?  
%	→ By registering live LiDAR scans to a pre-built 3D map, the system maintained global alignment and reduced drift significantly, even in the absence of loop closures or GNSS signals.
%	
%	\item \textbf{RQ3:} How does the proposed system maintain real-time performance and scalability in large-scale environments?  
%	→ The use of multithreaded NDT-OMP and a grid-based submap loading strategy enabled efficient scan-to-map matching, keeping per-frame processing time under 23 ms and limiting memory usage.
%	
%	\item \textbf{RQ4:} Does the proposed localization method maintain accuracy in feature-sparse and noisy environmental conditions?  
%	→ Yes. The system preserved localization accuracy in challenging scenarios, including fog, semi-dynamic environments, and map boundaries or unmapped transition zones. The fusion strategy maintained consistent pose estimation even when scan matching was weakened by poor feature correspondence.
%\end{itemize}
%
%Through these contributions, the thesis provides a comprehensive and extensible solution to the problem of robust, real-time localization in complex environments and establishes a strong foundation for future integration with additional sensing modalities such as camera or radar.
