\chapter{Discussion}
\label{ch:discu}

This study aimed to develop a robust, real-time LiDAR-inertial localization system that leverages prebuilt maps, LiDAR-Inertial Odometry (FAST-LIO2), and efficient scan matching, all fused through a sliding-window factor graph framework. The evaluation across multiple datasets and challenging conditions provides strong evidence of the system’s effectiveness. This chapter discusses the findings in relation to the research questions.


\subsection{Improving Accuracy and Robustness with Prebuilt Maps}

The integration of a prebuilt 3D map significantly enhances the robustness and accuracy of localization in GNSS-denied environments. By aligning real-time LiDAR scans with globally consistent map segments, the system reduces drift that typically accumulates in odometry-only methods. Experimental results from the Saxion and KITTI datasets show that the proposed system achieves centimeter- to decimeter-level translational RMSE, while Fast-LIO2 alone reaches up to 2–3 meters of drift.

Moreover, in GNSS-challenged environments where SLAM methods depend on loop closure, the proposed system offers an advantage by relying on a static map as a global reference. Even in KITTI Sequence 05, where loop closure is feasible, SLAM baselines such as KISS-ICP and MOLA exhibit maximum translational errors over 6 meters, while the proposed system maintains low error throughout. This demonstrates the superiority of map-based localization in scenarios where loop closures are sparse, unreliable, or delayed